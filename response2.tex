\documentclass{article}

\usepackage[pdftex]{graphicx}
%\usepackage{thumbpdf}
%\usepackage[naturalnames]{hyperref}
\usepackage{xr}
\externaldocument{paper}

\usepackage{bm, booktabs}
\usepackage{amsmath}%
\usepackage{amsfonts}%
\usepackage{amssymb}%
%\usepackage{amsthm}
\usepackage{mathrsfs}
%\usepackage{graphicx}
%\usepackage{setspace}
\usepackage{cite}
\usepackage{units}
	%nice looking units
%\usepackage{times}
\usepackage[normalem]{ulem}
%\usepackage{algorithmic}
%\usepackage[figure, vlined, linesnumbered]{algorithm2e}
\usepackage[vlined, linesnumbered]{algorithm2e}
%\usepackage{algorithm}

%%%%% set up the bibliography style
\bibliographystyle{../IEEEbst}
%\usepackage[square,comma,numbers]{natbib}

\newcommand{\fracpart}[1]{\left\langle #1 \right\rangle}
\newcommand{\sfracpart}[1]{\langle #1 \rangle}
\newcommand{\round}[1]{\lfloor #1 \rceil}
\newcommand{\floor}[1]{\lfloor #1 \rfloor}
\newcommand{\ceil}[1]{\lceil #1 \rceil}
\newcommand{\abs}[1]{{\left\vert #1 \right\vert}}
\newcommand{\sabs}[1]{{\vert #1 \vert}}
\newcommand{\reals}{{\mathbb R}}
\newcommand{\ints}{{\mathbb Z}}
\newcommand{\integers}{{\mathbb Z}}
\newcommand{\uy}{\underline{\bm{Y}}}
\newcommand{\uey}{\underline{Y}} 
\newcommand{\sign}[1]{\mathtt{sign}(#1)}
\newcommand{\Qbf}{{\mathbf Q}}
\newcommand{\Bbf}{{\mathbf B}}
\newcommand{\Ibf}{{\mathbf I}}
\newcommand{\ybf}{{\mathbf y}}
\newcommand{\xbf}{{\mathbf x}}
\newcommand{\zbf}{{\mathbf z}}
\newcommand{\ebf}{{\mathbf e}}
\newcommand{\vbf}{{\mathbf v}}
\newcommand{\wbf}{{\mathbf w}}
\newcommand{\kbf}{{\mathbf k}}
\newcommand{\sbf}{{\mathbf s}}
\newcommand{\tbf}{{\mathbf t}}
\newcommand{\pbf}{{\mathbf p}}
\newcommand{\Pibf}{{\mathbf \Pi}}
\newcommand{\onebf}{{\mathbf 1}}
\newcommand{\fbf}{{\mathbf f}}
\newcommand{\ubf}{{\mathbf u}}
\newcommand{\expect}{{\mathbb E}}

\newcommand{\NP}{\operatorname{NearestPt}}
\newcommand{\NS}{\operatorname{NearestSet}}
\newcommand{\bres}{\operatorname{Bres}}
\newcommand{\vol}{\operatorname{vol}}
\newcommand{\vor}{\operatorname{Vor}}
\newcommand{\coef}{\operatorname{Coef}}
\newcommand{\Int}{\operatorname{Int}}

\newtheorem{theorem}{Theorem}

\begin{document}


\section{Comments from reviewer 1}

\begin{enumerate}

\item \textbf{Comment:} Consistently write ``non-data-aided'' throughout the whole text \\
\textbf{Response:} Done.

\item \textbf{Comment:} Page (P) 42, right column (RC), line (L) 50: Aside from this extension of \dots \\
\textbf{Response:} Fixed.


\item \textbf{Comment:} P 43, LC, L 44: \dots convincing response \dots \\
\textbf{Response:} Fixed.

\item \label{eq:rev1:multantenna} \textbf{Comment:} P 43, RC, L 2+4: \dots as multiple antenna systems \dots must withstand rapidly \dots \\
\textbf{Response:} Fixed.

\item \textbf{Comment:} P 43, RC, L 21+28: \dots aided the design of \dots to other receiver components \dots \\
\textbf{Response:} Fixed.

\item \textbf{Comment:} Conference papers: location missing \\
\textbf{Response:} The location of conferences has been added.

\item \textbf{Comment:} Some journal papers: month of publication missing \\
\textbf{Response:} The month has been added whenever it was available in the original publication. 

\item \textbf{Comment:} [8] \dots IEEE Commun. Letters \dots \\
\textbf{Response:} Fixed.

\item \textbf{Comment:} [31] N. Noels, V. Lottici, \dots \\
\textbf{Response:} Fixed.

\end{enumerate}

\section{Comments from reviewer 2}

\begin{enumerate}

\item \textbf{Comment:} In the abstract: non data-aided $\to$ change to: non-data-aided \\
\textbf{Response:} Fixed.

\item \textbf{Comment:} ``we assume that the time-offset has been previously handled'' $\to$ perhaps some motivation for this assumption and the consequences with respect to the system model would be in place \\
\textbf{Response:}  The assumption of an accurate time offset estimator is common in this literature~\cite{ViterbiViterbi_phase_est_1983,Cowley_ref_sym_carr_1998,Wilson1989,Makrakis1990,Liu1991,Mackenthun1994,Sweldens2001,McKilliamLinearTimeBlockPSK2009,Divsalar1990} and is unlikely to come as a surprise to the intended audience.  There exist numerous estimators for the time offset in the literature.  We have added the sentence 

``Methods for estimating the time offset are given in~\cite{Massey1972optimumframe,Oerder_synch_square_circstat_2008,McKilliam_time_offset_pilots_data_2013}.'' 

to inform the reader of some of these methods.  The reader interested in the motivation and consequences of this assumption can consult this literature.


\item \textbf{Comment:} ``such as binary phase shift keying (BPSK), \dots'' $\to$ mentioning BPSK and QPSK apart from general M-PSK is pure overhead in this stage, I would drop it \\
\textbf{Response:} Done.

\item \textbf{Comment:} ``the data symbols are not known to the receiver and must also be estimated'' $\to$ I would drop the second part of the sentence because (e.g. Viterbi and Vterbi estimator) non-data-aided estimators typically do not actually estimate the unknown data symbols \\
\textbf{Response:} Done.

\item \textbf{Comment:} p. 3: ``bounds that are valid asymptoticially'': typo in asymptotically \\
\textbf{Response:} Fixed.

\item \textbf{Comment:} p. 3: ``all of the exiting bounds are applicable only in the fully non-data-aided case or in the completely data-aided case \dots'' $\to$ bounds for the common scenario where there is a mixture of pilot and data symbols can be found in ``N. Noels, H. Steendam, M. Moeneclaey en H. Bruneel, Carrier Phase and Frequency Estimation for Pilot Symbol Assisted Transmission: Bounds and Algorithms, IEEE TSP, Vol. 53, No. 12, p. 4578-4587, Dec. 2005.'' \\
\textbf{Response:} We have added this reference and have removed the sentences begining ``To the authors' knowledge, all of the existing bounds are applicable only in the fully non-data-aided case \dots''. Note that the model considered in the paper by Noels~et.~al.~\cite{Noels_crbs_with_pilots_2005} treats the amplitude as fixed and known and includes a frequency offset term.  Thus, the bounds are not immediately applicable in our setting and we are not able to meaningfully compare these bounds with our simulation results.  %For this reason with have kept with the bound described in~\cite{950345} as in the previous version of the manuscript.

\item \textbf{Comment:} p. 3: ``the situations is'' $\to$ drop `s' at the end of situations \\
\textbf{Response:} Fixed.

\item \textbf{Comment:} p. 3: in continuous text I prefer the use of ``approached infinity'' rather than ``$\to \infty$'' \\
\textbf{Response:} This change has been made wherever it occurs in the introduction, but not throughout the whole text. This is a personal preference of the authors.

\item \textbf{Comment:} p. 3: last line: typo in ``convincing'' \\
\textbf{Response:} Fixed.

\item \textbf{Comment:} p. 4: ``systems such as mulitple anntannae'' $\to$ should be ``multiple antennae'' \\
\textbf{Response:} Reviewer 1 comment~\eqref{eq:rev1:multantenna} wanted ``antenna'' and we have used this spelling.

\item \textbf{Comment:} p. 4: ``the other reciever components'' $\to$ should be ``receiver components'' \\
\textbf{Response:} Fixed.

\item \textbf{Comment:} p. 5 (and further throughout the text): ``pilots symbols" (with `s') and ``pilot symbols'' (without `s') are both used $\to$ use only 1, I prefer ``pilot symbols'' \\
\textbf{Response:} All occurences of ``pilots symbols'' have been changed to ``pilot symbols''. 

\item \label{rev2:commentnotationsection} \textbf{Comment:} I think it would be a good idea to specify the main notational conventions at the end of the introductory section \\
\textbf{Response:} We respectfully request not to make this change.  It is the authors' preference to introduce notation as it is used in the text rather than in a dedicated notation section. %BLERG. This need to be done. It's connected with comment~\ref{rev2:commentE}.

\item \textbf{Comment:} between (5) and (6): start new paragraph \\
\textbf{Response:} Done.  The paragraph ends at ``\dots least squares estimator of the data symbols~\cite{Sweldens2001,Mackenthun1994}'' and the new paragraph begins with ``To see this, let $a = \rho e^{j\theta}$ \dots''

\item \textbf{Comment:} before (6): drop the 'to' \\
\textbf{Response:} Done.

\item \textbf{Comment:} before (7): change ``for given $\theta$'' to ``for given $\theta$ and $\rho$'' \\
\textbf{Response:} Done.

\item \textbf{Comment:} what is the use of keeping the constants `A' and `B' in the expressions $\to$ can they not be dropped? \\
\textbf{Response:} The constants are retained so that statements of equality, such as
\[
LS(\{d_i, i \in D\}) = A - \frac{1}{L}\abs{Y}^2
\]
are valid. For the purpose of minimisation the constants have no effect.  However, in this case, we feel that introducing notation to describe ``equivalent for the purpose of minimisation'' is more cumbersome than simply retaining constants in the few cases that they occur.

\item \textbf{Comment:} I would prefer to see comma's between two subindices; so, `$k,i$' rather than`'$ki$' \\
\textbf{Response:} Done.

\item \textbf{Comment:} below (11) '$g_{ik}$' should be '$g_{ki}$' (change order of the subscripts) \\
\textbf{Response:} Fixed.

\item \label{rev2:commentE} \textbf{Comment:} I would prefer to see brackets around the functions of which the mean is computed (use of the operator $\mathbb E$) \\
\textbf{Response:} We prefer not to use brackets whenever there is no chance of ambiguity.  To clarify this for the reader the following sentences have been added to Section~\ref{sec:stat-prop-least}:

``Whenever there is no chance of ambiguity we write simply $\expect X$ rather than $\expect(X)$ to denote the expected value of the random variable $X$.  For two random variables $X$ and $Y$ the notation $\expect X Y$ will always mean the expected value of the product $XY$, that is, $\expect(XY)$ and not $\expect(X)Y$.''

The bracketless notation is heavily employed in the statistical literature on empirical processes~\cite{Pollard_new_ways_clts_1986,Pollard_conv_stat_proc_1984,Pollard_asymp_empi_proc_1989,van2009empirical}. The notation would become heavily cluttered otherwise.  Our proofs in the appendix rely upon this literature.

\item \textbf{Comment:} p. 12: ``the amplitude estimation error with bias removed'' $\to$ bit strange formulation \\
\textbf{Response:} This phrase has been removed.

\item \textbf{Comment:} p. 14: ``Equivalently $\hat\theta$ is the maximiser of \dots'' $\to$ could you please provide some further explanation (is the maximum of $\Re(Z(\theta))$ never negative? and why) \\
\textbf{Response:} The least squares phase estimator is the minimiser of
\[
LS(\theta) = A - \frac{1}{L}\Re(Z(\theta))^2.
\]
Because the least squares amplitude estimator $\hat{\rho} = \Re(Z(\hat{\theta}))$ is positive by definition, we are only interested in those values of $\theta$ that correspond with a positive amplitude estimator, that is, such that
\[
\hat{\rho}(\theta) = \Re(Z(\theta)) > 0.
\]
Thus, the least squares phase estimator $\hat{\theta}$ must be such that
\[
\hat{\rho} = \Re(Z(\hat{\theta})) > 0.
\]
This is explained in the text by the sentences:

``By definition the amplitude $\rho_0$ and its estimator $\hat{\rho}$ are positive.  However, $\hat{\rho}(\theta) = \Re(Z(\theta))$ may take negative values for some $\theta \in [-\pi,\pi)$.  The least square estimator $\hat{\theta}$ of $\theta_0$ is the minimiser of $LS(\theta)$ under the constraint $\hat{\rho}(\theta) = \Re(Z(\theta)) > 0$.  Equivalently $\hat{\theta}$ is the maximiser of $\Re(Z(\theta))$ with no constraints required.''

\item \textbf{Comment:} p. 15: the use of `A' as notation for a sample space is confusing since `A' has already been used as a constant \\
\textbf{Response:} $A$ has been replaced by $\Sigma$ and $A^\prime$ by $\Sigma^\prime$.

\item \textbf{Comment:} in general, I find the numerical results section rather short \\
\textbf{Response:} We have greatly expanded the simulation section as detailed in the responses that follow.

\item \textbf{Comment:} although different figures are shown for M=2,4,8 (phase estimation), the effect of the constellation size is not addressed in the text  \\
\textbf{Response:}  We have included the follows sentences to Section~\ref{sec:simulations}:

``As expected, the figures indicate the estimator to become more accurate as the number of pilot symbols is increased.  An interesting property distriguishing the case when no pilot symbols exist is the slope of the MSE curve at low SNR.  When a finite proportion of pilot symbols exist the slope at low SNR is similar to that when all symbols are pilots.  The slope is steeper in the case that no pilot symbols exist.''

\item \textbf{Comment:} p. 18, first sentence: ``when $L$ is sufficiently large'' $\to$ in these figures the value of $L$ is fixed, should this be ``when the SNR is sufficiently large''  \\
\textbf{Response:} Fixed.

\item \textbf{Comment:} p. 18, first sentence: a qualitative explanation for the fact the theory and simulations diverge at low SNR would be appreciated  \\
\textbf{Response:} BLERG: Added a reference to Tazras crb paper.

\item \textbf{Comment:} p.18, last sentence of first paragraph: ``non-data aided'' $\to$ should be ``non-data-aided''  \\
\textbf{Response:} Fixed.

\item \textbf{Comment:}  although the most important contribution of the paper is with respect to the biased amplitude estimate, only 1 figure is presented for amplitude estimation, namely for $L=2048$ and BPSK $\to$ this seems not well balanced  \\
\textbf{Response:} Figures have been added showing the amplitude estimation error for BSPK, QPSK, and 8-PSK.

\item \textbf{Comment:}  a separate plot of the two terms that contribute to Fig. 4 would be welcome \\

\textbf{Response:}

\end{enumerate}

\section{Comments from reviewer 3}

\begin{enumerate}

\item \textbf{Comment:}  Eq. (14) would be easier to follow with the intermediate step inserted $\dots = (a_0s_i + w_i)/(a_0s_i) = \dots$ \\
\textbf{Response:} Done.

\item \textbf{Comment:}  There are many formulas (starting from page 11) with the notation $\fracpart{x}$ meaning ``$x$ taken modulo $2\pi/M$''. These angle brackets don't strike the eye and can be easily confused with ordinary parentheses. To make them much more visible, the authors could use the same approach as for ``$x$ taken modulo $2\pi$'' - by using corresponding subscript, i.e. use $\fracpart{x}_{2\pi/M}$ instead of $\fracpart{x}$. \\
\textbf{Response:} We request not to make this change.  There are two reasons.  The first is that this notation leads to situations where there are subscripts withing superscripts.  For example, in the proof of Theorem 1 we would have expressions of the form
\[
\rho_0 R_i e^{j\sfracpart{\lambda + \Phi_i}_{2\pi/M}}
\]
and in the appendix we would obtain expressions such as
\[
2 \hat{\rho} \rho_0 R_k \Re\left( \eta e^{j\sfracpart{\hat{\lambda}_L + \Phi_k}_{2\pi/M}}\right)
\]
and
\[
q_2(\lambda) + j k_2(\lambda) = \expect\big[ R_1 e^{j\fracpart{\lambda + \Phi_1}_{2\pi/M}} - R_1 e^{j\fracpart{\Phi_1}_{2\pi/M}}  \big].
\]
We feel that these are more difficult to parse that the existing simpler
\[
\rho_0 R_i e^{j\sfracpart{\lambda + \Phi_i}},
\]
\[
2 \hat{\rho} \rho_0 R_k \Re\left( \eta e^{j\sfracpart{\hat{\lambda}_L + \Phi_k}}\right),
\]
and
\[
q_2(\lambda) + j k_2(\lambda) = \expect\big[ R_1 e^{j\fracpart{\lambda + \Phi_1}} - R_1 e^{j\fracpart{\Phi_1}}  \big].
\]

The second reason is that some of the authors have made use of the notation $\fracpart{\cdot}$ is previous work~\cite{McKilliamFrequencyEstimationByPhaseUnwrapping2009,McKilliam_mean_dir_est_sq_arc_length2010,McKilliam_pps_unwrapping_tsp_2014} and there is a desire for continuity in the notation between these papers.

 \end{enumerate}

\section{Comments from reviewer 4}

\begin{enumerate}

\item \textbf{Comment:}  The variable  ``a'' in the equation 2 is not defined in the text \\
\textbf{Response:} In~\eqref{eq:LSdefn} the variable $a$ is an argument of the function $LS$, that is,
\[
\begin{split}
LS(a, &\{d_i, i \in D\}) = \sum_{i \in P \cup D} \abs{ y_i - a s_i }^2  \\
&= \sum_{i \in P} \abs{ y_i - a p_i }^2 + \sum_{i \in D} \abs{ y_i - a d_i }^2.
\end{split}
\]

\item \textbf{Comment:}  There are typografical errors in the words  ``multiple antenna'' in line 2, page 2, section 1 \\
\textbf{Response:} Fixed.

\item \textbf{Comment:}  And also in the word ``receiver" in line 28 of the same page.
 \\
\textbf{Response:} Fixed.


\end{enumerate}

%\documentclass[a4paper,10pt]{article}
%\documentclass[draftcls, onecolumn, 11pt]{IEEEtran}
\documentclass[conference]{IEEEtran}

\usepackage{mathbf-abbrevs}
%\newcommand {\tbf}[1] {\textbf{#1}}
%\newcommand {\tit}[1] {\textit{#1}}
%\newcommand {\tmd}[1] {\textmd{#1}}
%\newcommand {\trm}[1] {\textrm{#1}}
%\newcommand {\tsc}[1] {\textsc{#1}}
%\newcommand {\tsf}[1] {\textsf{#1}}
%\newcommand {\tsl}[1] {\textsl{#1}}
%\newcommand {\ttt}[1] {\texttt{#1}}
%\newcommand {\tup}[1] {\textup{#1}}
%
%\newcommand {\mbf}[1] {\mathbf{#1}}
%\newcommand {\mmd}[1] {\mathmd{#1}}
%\newcommand {\mrm}[1] {\mathrm{#1}}
%\newcommand {\msc}[1] {\mathsc{#1}}
%\newcommand {\msf}[1] {\mathsf{#1}}
%\newcommand {\msl}[1] {\mathsl{#1}}
%\newcommand {\mtt}[1] {\mathtt{#1}}
%\newcommand {\mup}[1] {\mathup{#1}}

%some math functions and symbols
\newcommand{\reals}{{\mathbb R}}
\newcommand{\expect}{{\mathbb E}}
\newcommand{\indicator}{{\mathbf 1}}
\newcommand{\ints}{{\mathbb Z}}
\newcommand{\complex}{{\mathbb C}}
\newcommand{\integers}{{\mathbb Z}}
\newcommand{\sign}[1]{\mathtt{sign}\left( #1 \right)}
\newcommand{\NP}{\operatorname{NPt}}
\newcommand{\erf}{\operatorname{erf}}
\newcommand{\NS}{\operatorname{NearestSet}}
\newcommand{\bres}{\operatorname{Bres}}
\newcommand{\vol}{\operatorname{vol}}
\newcommand{\vor}{\operatorname{Vor}}
%\newcommand{\re}{\operatorname{Re}}
%\newcommand{\im}{\operatorname{Im}}




\newcommand{\term}{\emph}
\newcommand{\var}{\operatorname{var}}
%\newcommand{\prob}{{\mathbb P}}
\newcommand{\prob}{{\operatorname{Pr}}}

%distribution fucntions
\newcommand{\projnorm}{\operatorname{ProjectedNormal}}
\newcommand{\vonmises}{\operatorname{VonMises}}
\newcommand{\wrapnorm}{\operatorname{WrappedNormal}}
\newcommand{\wrapunif}{\operatorname{WrappedUniform}}

\newcommand{\selectindicies}{\operatorname{selectindices}}
\newcommand{\sortindicies}{\operatorname{sortindices}}
\newcommand{\largest}{\operatorname{largest}}
\newcommand{\quickpartition}{\operatorname{quickpartition}}
\newcommand{\quickpartitiontwo}{\operatorname{quickpartition2}}

%some commonly used underlined and
%hated symbols
\newcommand{\uY}{\ushort{\mbf{Y}}}
\newcommand{\ueY}{\ushort{Y}}
\newcommand{\uy}{\ushort{\mbf{y}}}
\newcommand{\uey}{\ushort{y}}
\newcommand{\ux}{\ushort{\mbf{x}}}
\newcommand{\uex}{\ushort{x}}
\newcommand{\uhx}{\ushort{\mbf{\hat{x}}}}
\newcommand{\uehx}{\ushort{\hat{x}}}

% Brackets
\newcommand{\br}[1]{{\left( #1 \right)}}
\newcommand{\sqbr}[1]{{\left[ #1 \right]}}
\newcommand{\cubr}[1]{{\left\{ #1 \right\}}}
\newcommand{\abr}[1]{\left< #1 \right>}
\newcommand{\abs}[1]{{\left\vert #1 \right\vert}}
\newcommand{\sabs}[1]{{\vert #1 \vert}}
\newcommand{\floor}[1]{{\left\lfloor #1 \right\rfloor}}
\newcommand{\ceiling}[1]{{\left\lceil #1 \right\rceil}}
\newcommand{\ceil}[1]{\left\lceil #1 \right\rceil}
\newcommand{\round}[1]{{\left\lfloor #1 \right\rceil}}
\newcommand{\magn}[1]{\left\| #1 \right\|}
\newcommand{\fracpart}[1]{\left\langle #1 \right\rangle}
\newcommand{\sfracpart}[1]{\langle #1 \rangle}


% Referencing
\newcommand{\refeqn}[1]{\eqref{#1}}
\newcommand{\reffig}[1]{Figure~\ref{#1}}
\newcommand{\reftable}[1]{Table~\ref{#1}}
\newcommand{\refsec}[1]{Section~\ref{#1}}
\newcommand{\refappendix}[1]{Appendix~\ref{#1}}
\newcommand{\refchapter}[1]{Chapter~\ref{#1}}

\newcommand {\figwidth} {100mm}
\newcommand {\Ref}[1] {Reference~\cite{#1}}
\newcommand {\Sec}[1] {Section~\ref{#1}}
\newcommand {\App}[1] {Appendix~\ref{#1}}
\newcommand {\Chap}[1] {Chapter~\ref{#1}}
\newcommand {\Lem}[1] {Lemma~\ref{#1}}
\newcommand {\Thm}[1] {Theorem~\ref{#1}}
\newcommand {\Cor}[1] {Corollary~\ref{#1}}
\newcommand {\Alg}[1] {Algorithm~\ref{#1}}
\newcommand {\etal} {\emph{~et~al.}}
\newcommand {\bul} {$\bullet$ }   % bullet
\newcommand {\fig}[1] {Figure~\ref{#1}}   % references Figure x
\newcommand {\imp} {$\Rightarrow$}   % implication symbol (default)
\newcommand {\impt} {$\Rightarrow$}   % implication symbol (text mode)
\newcommand {\impm} {\Rightarrow}   % implication symbol (math mode)
\newcommand {\vect}[1] {\mathbf{#1}} 
\newcommand {\hvect}[1] {\hat{\mathbf{#1}}}
\newcommand {\del} {\partial}
\newcommand {\eqn}[1] {Equation~(\ref{#1})} 
\newcommand {\tab}[1] {Table~\ref{#1}} % references Table x
\newcommand {\half} {\frac{1}{2}} 
\newcommand {\ten}[1] {\times10^{#1}}
\newcommand {\bra}[2] {\mbox{}_{#2}\langle #1 |}
\newcommand {\ket}[2] {| #1 \rangle_{#2}}
\newcommand {\Bra}[2] {\mbox{}_{#2}\left.\left\langle #1 \right.\right|}
\newcommand {\Ket}[2] {\left.\left| #1 \right.\right\rangle_{#2}}
\newcommand {\im} {\mathrm{Im}}
\newcommand {\re} {\mathrm{Re}}
\newcommand {\braket}[4] {\mbox{}_{#3}\langle #1 | #2 \rangle_{#4}} 
\newcommand{\dotprod}[2]{ \left\langle #1 , #2 \right\rangle}
\newcommand {\trace}[1] {\text{tr}\left(#1\right)}

% spell things correctly
\newenvironment{centre}{\begin{center}}{\end{center}}
\newenvironment{itemise}{\begin{itemize}}{\end{itemize}}

%%%%% set up the bibliography style
\bibliographystyle{IEEEbib}
%\bibliographystyle{uqthesis}  % uqthesis bibliography style file, made
			      % with makebst

%%%%% optional packages
\usepackage[square,comma,numbers,sort&compress]{natbib}
		% this is the natural sciences bibliography citation
		% style package.  The options here give citations in
		% the text as numbers in square brackets, separated by
		% commas, citations sorted and consecutive citations
		% compressed 
		% output example: [1,4,12-15]

%\usepackage{cite}		
			
\usepackage{units}
	%nice looking units
		
\usepackage{booktabs}
		%creates nice looking tables
		
\usepackage{ifpdf}
\ifpdf
  \usepackage[pdftex]{graphicx}
  %\usepackage{thumbpdf}
  %\usepackage[naturalnames]{hyperref}
\else
	\usepackage{graphicx}% standard graphics package for inclusion of
		      % images and eps files into LaTeX document
\fi

\usepackage{amsmath,amsfonts,amssymb, amsthm, bm} % this is handy for mathematicians and physicists
			      % see http://www.ams.org/tex/amslatex.html

		 
\usepackage[vlined, linesnumbered]{algorithm2e}
	%algorithm writing package
	
\usepackage{mathrsfs}
%fancy math script

%\usepackage{ushort}
%enable good underlining in math mode

%------------------------------------------------------------
% Theorem like environments
%
\newtheorem{theorem}{Theorem}
%\theoremstyle{plain}
\newtheorem{acknowledgement}{Acknowledgement}
%\newtheorem{algorithm}{Algorithm}
\newtheorem{axiom}{Axiom}
\newtheorem{case}{Case}
\newtheorem{claim}{Claim}
\newtheorem{conclusion}{Conclusion}
\newtheorem{condition}{Condition}
\newtheorem{conjecture}{Conjecture}
\newtheorem{corollary}{Corollary}
\newtheorem{criterion}{Criterion}
\newtheorem{definition}{Definition}
\newtheorem{example}{Example}
\newtheorem{exercise}{Exercise}
\newtheorem{lemma}{Lemma}
\newtheorem{notation}{Notation}
\newtheorem{problem}{Problem}
\newtheorem{proposition}{Proposition}
\newtheorem{property}{Property}
\newtheorem{remark}{Remark}
\newtheorem{solution}{Solution}
\newtheorem{summary}{Summary}
%\numberwithin{equation}{section}
%--------------------------------------------------------


\usepackage{amsmath,amsfonts,amssymb, amsthm, bm}

\usepackage[square,comma,numbers,sort&compress]{natbib}

\newcommand{\sgn}{\operatorname{sgn}}
\newcommand{\sinc}{\operatorname{sinc}}
\newcommand{\rect}[1]{\operatorname{rect}\left(#1\right)}

%opening
\title{Noncoherent least squares estimators of carrier phase and amplitude}
\author{Robby McKilliam$^{1}$, Andre Pollok$^{1}$, Bill Cowley$^{1}$, Vaughan Clarkson$^{2}$ and Barry Quinn$^{3}$ \vspace{0.2cm} \\
\small $^{1}$Institute for Telecommunications Research, University of South Australia, SA, \textsc{Australia} \vspace{0.1cm} \\
\small $^{2}$School of Information Technology \& Electrical Engineering, The University of Queensland, QLD 4067, \textsc{Australia} \vspace{0.1cm} \\
\small $^{3}$Department of Statistics, Macquarie University, Sydney, \textsc{Australia} \\}


\begin{document}

\maketitle

\begin{abstract}
We consider least squares estimators of carrier phase and amplitude from a noisy communications signal.  We focus on signaling constellations that have symbols evenly distributed on the complex unit circle, such as binary phase shift keying (BPSK), quaternary phase shift keying (QPSK), and $M$-PSK.  We show, under reasonably mild conditions on the distribution of the noise, that the least squares estimator of carrier phase is strongly consistent and asymptotically normaly distributed.  However, the amplitude estimator is not consistent, and we show that it converges to a positive real number that is a function of the true carrier amplitude, the noise distribution, and the size of the constellation.  The results of Monte Carlo simulations are provided and these corroborate the theoretical results.
\end{abstract}
\begin{IEEEkeywords}
BLERG
\end{IEEEkeywords}

\section{Introduction}

In passband communications systems the transmitted signal typically undergoes time offset (delay), phase shift and attenuation.  These effects must be compensated for at the reciever. In this paper we assume that the time offset has been previously handled, and we focus on estimating the phase shift and attenuation.  We consider signalling constellations that have symbols evenly distributed on the complex unit circle, commonly refered to as $M$-ary phase shift keying ($M$-PSK).  In this case, the transmitted symbols take the form,
\[
s_i = e^{j u_i},
\]
where $j = \sqrt{-1}$ and $u_i$ is from the set $\{0, \tfrac{2\pi}{M}, \dots, \tfrac{2\pi(M-1)}{M}\}$ and $M$ is the size of the constellation.

We assume that time offset estimation and matched filtering have already been performed at the reciever.  The recieved symbols then take the form,
\begin{equation}\label{eq:sigmod}
y_i = a_0 s_i + w_i, \qquad i = 1, \dots, L,
\end{equation}
where $w_i$ is noise and $a_0 = \rho_0 e^{j\theta_0}$ is a complex number representing both carrier phase $\theta_0$ and amplitude $\rho_0$ (by definition $\rho_0$ is a positive real number).  Our aim is to estimate $a_0$ from $y_1, \dots, y_L$.  Complicating matters is that the transmitted symbols $s_i$ are not known to the reciever and must also be estimated.  Estimation problems of this type have undergone extensive prior study~\cite{ViterbiViterbi_phase_est_1983,Cowley_ref_sym_carr_1998,Wilson1989,Makrakis1990,Liu1991,Mackenthun1994,Sweldens2001}.  A practical approach is the least squares estimator, that is, the minimisers of the sum of squares function
\[
SS(a, \{s_i, i \in D\}) = \sum_{i = 1}^L \abs{ y_i - a s_i }^2 
\]
where $\abs{x}$ denotes the magnitude of the complex number $x$.  The least squares estimator is also the maximum likelihood estimator under the assumption that the noise sequence $\{w_i, i \in \ints\}$ is additive white and Gaussian.  As we will show, the estimator works well under less stringent assumptions.  Mackenthun~\cite{Mackenthun1994} described an algorithm to compute the least squares estimator that requires only $O(L \log L)$ arithmetic operations.  Sweldens~\cite{Sweldens2001} rediscovered Mackenthun's algorithm in 2001.

% An alternative approach is the so called \emph{non-data aided}, sometimes also called \emph{non-decision directed}, estimator based on the paper of Viterbi and Viterbi~\cite{ViterbiViterbi_phase_est_1983}.  The idea is the `strip' the modulation from the recieved signal by taking $y_i / \abs{y_i}$ to the power of $M$.  A function $F$, mapping $\reals$ to $\reals$, is chosen and the estimator of the carrier phase $\theta_0$ is taken to be $\tfrac{1}{M}\angle{A}$ where $\angle$ denotes the complex argument and $A$ is the average
% \[
% A = \frac{1}{L}\sum_{i \in P \cup D} F(\abs{y_i}) \big(\tfrac{y_i}{\abs{y_i}}\big)^M.
% \]
% Various choices for $F$ are suggested in~\cite{ViterbiViterbi_phase_est_1983} and a statistical analysis is presented.

In the literature it has been common to assume that the symbols $\{s_i, i \in D\}$ are of primary interest and the complex amplitude $a_0$ is a nuisance parameter.  The metric of performance is correspondingly \emph{symbol error rate}, or \emph{bit error rate}.  Whilst estimating the symbols (or more precisely the transmitted bits) is ultimately the goal, we take the opposite approach here.  Our aim is to estimate $a_0$, and we treat the unknown symbols as nuisance parameters.  This is motivated by the fact that in many modern communication systems the data symbols are \emph{coded}.  For this reason raw symbol error rate is not of specific interest at this stage.  Instead, we desire an accurate estimator $\hat{a}$ of $a_0$, so that the compensated recieved symbols $\hat{a}^{-1}y_i$ can be accurately modelled using an additive noise channel.  The additive noise channel is a common assumption for subsequent reciever operations, such as decoding.  Consequently, our metric of performance will not be symbol or bit error rate, it will be mean square error (MSE) between complex amplitude $a_0$ and its estimator $\hat{a}$, that is, our metric is $\expect\abs{a_0 - \hat{a}}^2$ where $\expect$ denotes the expected value. It will actually be informative to consider the carrier phase and amplitude estimators separately, that is, if $\hat{a} = \hat{\rho}e^{j\hat{\theta}}$ where $\hat{\rho}$ is a positive real number, then we consider $\expect\langle\theta_0 - \hat{\theta}\rangle^2$ and $\expect(\rho_0 - \hat{\rho})^2$.  The function $\fracpart{\cdot}$ denotes its argument taken `modulo $\tfrac{2\pi}{M}$' into the interval $[-\tfrac{\pi}{M}, \tfrac{\pi}{M})$.  It will become apparent why $\expect\langle\theta_0 - \hat{\theta}\rangle^2$ rather than $\expect(\theta_0 - \hat{\theta})^2$ is the appropriate measure of error for the phase parameter.

%It is worth commenting on our use of $\prob$ rather than the more usual $\expect$ or $E$ for the expected value operator.  The notation comes from Pollard~\cite[Ch 1]{Pollard_users_guide_prob_2002}.  The notation is good because it removes unecessary distinction between `probability' and expectation.  Given a random variable $X$ with cumulative density function $F$, the probability of an event, say $X \in S$, where $S$ is some subset of the range of $X$, is 
%\[
%\prob \indicator \{X \in S\} = \int \indicator \{X \in S\}(x) dF(x) = \int_{S} dF(x)
%\]
%where $\indicator \{X \in S\}$ is the indicator function of the set $S$, i.e $\indicator \{X \in S\}(x) = 1$ when the argument $x \in S$ and zero otherwise.  We will usually drop the $\onebf$ and simply write $\prob \{ X \in S \}$ to mean $\prob \onebf\{ X \in S \}$.  To illustrate the utility of this notation, Markov's inequality becomes 
%\[
%\prob \{ \abs{X} > \delta \}  \leq \prob \frac{\abs{X}}{\delta}\onebf\{ \abs{X} > \delta \} \leq \frac{1}{\delta}\prob\abs{X},
%\]
%where $\frac{\abs{X}}{\delta}\onebf\{ \abs{X} > \delta \}(x)$ is the function equal to $\abs{x}/\delta$ when the argument $x > \delta$ and zero otherwise.


\section{The least squares estimator}\label{sec:least-squar-estim}

As discussed, the least squares estimator is given by the minimiser of $SS(a, \{s_i, i \in D\})$ over all $a \in \complex$ and $s_1, \dots, s_L$ in the $M$-PSK constellation.  For fixed $a$ the least squares estimators of $s_1, \dots, s_L$, as functions $a = \rho e^{j \theta}$, are
\[
\hat{s}_i(a) = e^{j\hat{u}_i(\theta)} \qquad \text{where} \qquad \hat{u}_i(\theta) = \round{\angle( e^{-j\theta}y_i)},
\]
where $\angle(\cdot)$ denotes the complex argument (or phase), and $\round{\cdot}$ rounds its argument to the nearest multiple of $\frac{2\pi}{M}$.  Substituting $\hat{s}_i(a)$ into $SS(a, \{s_i, i \in D\})$, give the sum of squares, conditioned on minimisation with respect to the transmitted symbols,
\[
SS(a) = \sum_{i = 1}^L \abs{ y_i - a \hat{s}_i(a) }^2.
\]
The least squares estimator of the carrier phase $\hat{\theta}$ and amplitude $\hat{\rho}$ then satisfy $\hat{a} = \hat{\rho}e^{j\hat{\theta}}$ where $\hat{a} = \arg\min_{a \in \complex} SS(a)$.  Mackenthun~\cite{Mackenthun1994} described an algorithm that computes $\hat{a}$ using $O(L\log L)$ operations. 



\section{Circularly symmetric complex random variables}\label{sec:circ-symm-compl}

Before describing the statistical properties of the least squares estimator, we first require some properties of complex valued random variables.  A complex random variable $X$ is said to be \emph{circularly symmetric} if the distribution of its phase $\angle{X}$ is uniform on $[0,2\pi)$ and is independent of the distribution of its magnitude $\abs{X}$.  That is, if $Z \geq 0$ and $\Theta \in [0,2\pi)$ are real random variables such that $Ze^{j\Theta} = X$, then $\Theta$ is uniformly distributed on $[0,2\pi)$ and is indepedent of $Z$.  If the probability density function (pdf) of $Z$ is $f_Z(z)$, then the joint pdf (pdf) of $\Theta$ and $Z$ is 
\[
f_{Z,\Theta}(z,\theta) = \frac{1}{2\pi}f_Z(z).
\]  
Observe that for any real number $\phi$, the distribution of $X$ is that same as that of $e^{j\phi}X$.  %If $\expect\abs{X} = \expect Z$ is finite, then $X$ has zero mean because
% \begin{align*}
% \expect X &= \int_{0}^{2\pi} \int_{0}^\infty z e^{j\theta} f_{Z,\Theta}(z,\theta) dz d\theta \\
% &= \frac{1}{2\pi} \int_{0}^{2\pi} e^{j\theta} \int_{0}^\infty z f_Z(z) dz d\theta \\
% &= \frac{1}{2\pi}\expect Z \int_{0}^{2\pi} e^{j\theta} d\theta = 0.
% \end{align*}

We will have particular use of complex random variables of the form $1 + X$ where $X$ is circularly symmetric.  Let $R \geq 0$ and $\Phi \in [0,2\pi)$ be real random variables satisfying, 
\[
R e^{j\Phi} = 1 + X.
\]
The joint distribution of $R$ and $\Phi$ can be shown to be
\[
f(r,\phi) = \frac{r f_Z(\sqrt{r^2 - 2r\cos\phi + 1})}{\sqrt{r^2 - 2r\cos\phi + 1}}.
\]
% The mean of $R e^{j\Phi}$ is equal to one because the mean of $X$ is zero.  So,
% \[
% \expect \Re(R e^{j\Phi}) = \expect R \cos(\Phi) = 1,
% \]
% where $\Re(\cdot)$ denotes the real part, and
% \[
% \expect \Im(R e^{j\Phi}) = \expect R \sin(\Phi) = 0,
% \]
% where $\Im(\cdot)$ denotes the imaginary part.  %The next lemma will be useful for the analysis of the least squares estimator.

% \begin{lemma}\label{lem:h1minedcircsym}
% Let $X = Z e^{j\Theta}$ be a circularly symmetric complex random variable with pdf $\tfrac{1}{2\pi}f_Z(z)$.  Let $R > 0$ and $\Phi \in [0, 2\pi)$ be random variables satisfying $R e^{j\Phi} = 1 + X$.  Let
% \[
% h_1(x) = \expect R \cos(x + \Phi).
% \]
% Then $h_1(x)$ is uniquely maximised at $0$ over the interval $[-\pi,\pi]$.
% \end{lemma}

% Before we begin the proof note that the requirement for $z^{-1}f_{Z}(z)$ to be non increasing implies that the probability density function of $Z e^{j \Theta}$ decreases as we move away from the origin. That is, the pdf of $Z e^{j \Theta}$ in rectangular coordinates is given by $z^{-1}f_{Z}(z)$, where $z = \sqrt{x^2 + y^2}$ and $x$ and $y$ are the real and imaginary parts of $Z e^{j \Theta}$, and this pdf is non increasing with $z$.  For example, the zero mean complex Gaussian distribution with independent real and imaginary parts satisfies this requirement.

% By the phrase ``$h_1(x)$ is uniquely maximised at $0$ over the interval $[-\pi,\pi]$'' it is meant that for any $\delta > 0$ there exists an $\epsilon > 0$ such that for those $x \in [-\pi, \pi]$, if 
% \[
% \abs{x} > \delta \qquad \text{then} \qquad  h_1(0) - h_2(x) > \epsilon.
% \]
% We will have further use of this definition in Section~\ref{sec:stat-prop-least}.  We are now ready to prove Lemma~\ref{lem:h1minedcircsym}.

% \begin{IEEEproof}
% BLERG
% \end{IEEEproof}


\section{Statistical properties of the least squares estimator}\label{sec:stat-prop-least}

The next two theorems decribe the asymptotic properties of the least squares estimator.  We ommit the proofs due to space constraints.  Proofs will be provided in an upcomming paper.

\begin{theorem}\label{thm:consistency} (Almost sure convergence)
Let $\{w_i\}$ be a sequence of independent and identically distributed, circularly symmetric complex random variables with $\expect \abs{w_1}$ finite, and let $y_1,\dots, y_L$ be given by~\eqref{eq:sigmod}.   Let $\hat{a} = \hat{\rho}e^{j\hat{\theta}}$ be the least squares estimator of $a_0 = \rho_0e^{j\theta_0}$.  Let $R_i \geq 0$ and $\Phi_i \in [0,2\pi)$ be real random variables satisfying
\begin{equation}\label{eq:RiandPhii}
R_ie^{j\Phi_i} = 1 + \frac{1}{\rho_0} w_i e^{-j\theta_0} s_i^*,
\end{equation}
and define the continuous function
\[
G(x) = \expect R_1 \cos\sfracpart{ x + \Phi_1}.
\]
If $\abs{G(x)}$ is uniquely maximised at $x = 0$ over the interval $[-\pi,-\pi)$, then:
\begin{enumerate}
\item $\sfracpart{\theta_0 - \hat{\theta}} \rightarrow 0$ almost surely as $L \rightarrow \infty$,
\item $\hat{\rho} \rightarrow \rho_0 G(0)$ almost surely as $L \rightarrow \infty$.
\end{enumerate}
\end{theorem}

\begin{theorem}\label{thm:normality} (Asymptotic normality)
Under the same conditions as Theorem~\ref{thm:consistency}, let $f(r,\phi)$ be the joint probability density function of $R_1$ and $\Phi_1$, and let
\[
g(\phi) = \int_{0}^{\infty} r f(r,\phi) dr.
\]
Put $\hat{\lambda}_L = \sfracpart{\theta_0 - \hat{\theta}}$ and $\hat{m}_L = \hat{\rho} - \rho_0 G(0)$. %If $g(\phi)$ is continuous at $\phi = \tfrac{2\pi}{M}k+\tfrac{\pi}{M}$ for each $k = 0, 1, \dots M-1$, then 
Then the distribution of $(\sqrt{L}\lambda_L, \sqrt{L}m_L)$ converges to the bivariate normal with zero mean and covariance
\[
\left( \begin{array}{cc} 
A H^{-2} & 0 \\
0 & B
\end{array} \right)
\]
as $L \rightarrow \infty$, where
\[
H = G(0) -  2 \sin(\tfrac{\pi}{M}) \sum_{k = 0}^{M-1} g(\tfrac{2\pi}{M}k + \tfrac{\pi}{M}),
\]
\[
A = \expect R_1^2\sin^2\fracpart{\Phi_1}, \;\;\; B = \expect R_1^2 \cos^2\fracpart{\Phi_1}. 
\]
\end{theorem}

\begin{corollary}
The mean square error of the amplitude estimator $\hat{\rho}$ is $\expect (\rho_0 - \hat{\rho})^2 = \frac{B}{L} + \rho_0^2(1 - G(0))^2$.
\end{corollary}

%BLERG discuss assumptions.  We have that $h_1$ is uniquely maximised at $0$ over the interval $[-\pi,\pi)$ since $\{w_i\}$ has zero mean.  The requirement that $h_2$ is uniquely maximised at 0 is stronger that we need.  We actually only require that the sum $p h_1(x) + dh_2(x)$ is maximised uniquely at 0 over the interval $[-\pi, \pi)$.  However, we have stated the theorem in this stronger form as it makes the proof in the noncoherent case (i.e. when there are no pilot symbols) follow more transparently.

%BLERG $\hat{a}$ is a biased estimator, particularly when the SNR is low, we show firs that $\hat{\theta}$ is strongly consistent, we are then able to analyse the behaviour of $\hat{a}$.

We now make some discussion regarding the assumptions made by these theorems. 

\begin{lemma}
Let $X = Z e^{j\Theta}$ be a circularly symmetric complex random variable with pdf $\tfrac{1}{2\pi}f_Z(z)$ such that $z^{-1} f_Z(z)$ is nonincreasing with $z$.  Let $R > 0$ and $\Phi \in [0, 2\pi)$ be random variables satisfying $R e^{j\Phi} = 1 + X$.  Let,
\[
G(x) = \expect R \cos\sfracpart{x + \Phi}.
\]
Then $G(x)$ is uniquely maximised at $x=0$ over the interval $[-\tfrac{\pi}{M},\tfrac{\pi}{M})$.
\end{lemma}
\begin{IEEEproof}
BLERG
\end{IEEEproof}


\section{The Gaussian noise case}

Let the noise sequence $\{w_i\}$ be complex Gaussian with independent real and imaginary parts having zero mean and variance $\sigma^2$.  The joint pdf of the real and imaginary parts is then
\[
p(x,y) = \frac{1}{2\pi\sigma^2}e^{-(x^2 + y^2)/\sigma^2}
\]
Theorem~\ref{thm:consistency}~and~\ref{thm:normality} hold and, since the distribution of $w_1$ is circularly symetric, the distribution of $R_1e^{j\Phi_1}$ is identical to the distribution of $1 + \frac{1}{\rho_0} w_1$.
%and the joint density function of the real and imaginary parts of $Z$ is
%\[
%p_Z(x,y) = \rho_0^2 p( \rho_0 (x - 1), \rho_0 y ) = \frac{\kappa^2}{2\pi}e^{-\kappa^2(x ^2 - 2x + 1 + y^2) }
%\] 
%where $\kappa = \tfrac{\rho_0}{\sigma}$.  
It can be shown that
\[
g(\phi) = \frac{\cos^2(\phi)}{4\sqrt{\pi}}\left( \frac{e^{-\kappa^2} }{\sqrt{\pi}} + \kappa \Psi(\phi)  e^{-\kappa^2\sin^2(\phi)}\cos(\phi) \right)
\]
where
\[
\Psi(\phi) = 1 + \erf(\kappa \cos(\phi)) = 1 + \frac{2}{\sqrt{\pi}}\int_0^{\kappa \cos(\phi)} e^{-t^2} dt .
\]
The value of $A$ can be computed by numerical integration.


\section{Simulations}\label{sec:simulations}


\begin{figure}[p]
	\centering
		\includegraphics[width=\linewidth]{code/data/plotncM2-2.mps}
		\caption{Phase error for BPSK}
		\label{fig:plotphase}
\end{figure}

\begin{figure}[p]
	\centering
		\includegraphics[width=\linewidth]{code/data/plotncM4-2.mps}
		\caption{Phase error for QPSK}
		\label{fig:plotphase}
\end{figure}

\begin{figure}[p]
	\centering
		\includegraphics[width=\linewidth]{code/data/plotncM2-1.mps}
		\caption{Amplitude error for BPSK}
		\label{fig:plotphase}
\end{figure}

\begin{figure}[p]
	\centering
		\includegraphics[width=\linewidth]{code/data/plotncM4-1.mps}
		\caption{Amplitude error for QPSK}
		\label{fig:plotphase}
\end{figure}

%\begin{figure}[tp]
%	\centering
%		\includegraphics[width=\linewidth]{code/data/plotncM8-2.mps}
%		\caption{Phase error for 8PSK}
%		\label{fig:plotphase}
%\end{figure}


\small
\bibliography{bib}


\end{document}
 

\end{document}
